\documentclass{ujarticle}

\usepackage{epsfig}
%read ascmac -> algorithm2e 
\usepackage{ascmac}
%\usepackage{algorithm2e}
%\usepackage[linesnumbered]{algorithm2e}
%\usepackage[linesnumbered,ruled]{algorithm2e}
\usepackage[linesnumbered,ruled,vlined]{algorithm2e}
\usepackage{amsmath, amssymb}
\title{Nigel Smart, Cryptography: An Introductionを読む}
\author{@kaito\_tateyama}

\begin{document}

\maketitle

\section*{Chapter 1}
\begin{itemize}
    \item
      Euclidean Algorithm
    \item
      Extended Euclidean Algorithm
    \item
      Chinese Remainder Thorem (CRT) Garner's Algorithm
    \item
      Shank's Algorithm
\end{itemize}

\section*{Euclidean Algorithm}
整数\(a,b\)に対し、以下のように最大公約数を定義する。

\begin{screen}
  {\bf 定義} 整数 \(a,b\) の最大公約数 \(\gcd(a,b)\) は \(a \neq 0\) または \(b \neq 0\) のときに以下のように定義される:
  \( \gcd(a,b) = \max(\{g \bigm| g \mid a \land g \mid b \}) \)
\end{screen}

\begin{algorithm}
\DontPrintSemicolon % Some LaTeX compilers require you to use \dontprintsemicolon instead
\KwIn{Two integers $a,b (a > b)$}
\KwOut{Greatest Common Divisor of $a,b$}
  {\bf algorithm} gcd {\bf is}:\;
  $a^{\prime} \gets b$\;
  $b^{\prime} \gets (a \mod b)$\;
  \Return{$\gcd(a^{\prime}, b^{\prime})$}\;
\caption{Euclidean Algorithm(recursion)}
\label{algo:max}
\end{algorithm}

このアルゴリズムは与えられた $a,b$ に対して必ず停止し、その時間計算量は $O(?)$である。以降、このアルゴリズムの停止性、計算量、正当性を示す。

\begin{algorithm}
\DontPrintSemicolon % Some LaTeX compilers require you to use \dontprintsemicolon instead 
\KwIn{A set $C = \{c_1, c_2, \ldots, c_r\}$ of denominations of coins, where $c_i > c_2 > \ldots > c_r$ and a positive number $n$}
\KwOut{A list of coins $d_1,d_2,\ldots,d_k$, such that $\sum_{i=1}^k d_i = n$ and $k$ is minimized}
$C \gets \emptyset$\;
\For{$i \gets 1$ \textbf{to} $r$}{
  \While{$n \geq c_i$} {
    $C \gets C \cup \{c_i\}$\;
    $n \gets n - c_i$\;
  }
}
\Return{$C$}\;
\caption{{\sc Change} Makes change using the smallest number of coins}
\label{algo:change}
\end{algorithm}

Algorithm~\ref{algo:duplicate} and Algorithm~\ref{algo:duplicate2} will
find the first duplicate element in a sequence of integers.

\begin{algorithm}
\DontPrintSemicolon % Some LaTeX compilers require you to use \dontprintsemicolon instead
\KwIn{A sequence of integers $\langle a_1, a_2, \ldots, a_n \rangle$}
\KwOut{The index of first location witht he same value as in a previous location in the sequence}
$location \gets 0$\;
$i \gets 2$\;
\While{$i \leq n$ \textbf{and} $location = 0$}{
  $j \gets 1$\;
  \While{$j < i$ \textbf{and} $location = 0$}{
    % The "u" before the "If" makes it so there is no "end" after the statement, so the else will then follow
    \uIf{$a_i = a_j$}{
      $location \gets i$\;
    }
    \Else{
      $j \gets j + 1$\;
    }
  }
  $i \gets i + 1$\;
}
\Return{location}\;
\caption{{\sc FindDuplicate}}
\label{algo:duplicate}
\end{algorithm}

\begin{algorithm}
\DontPrintSemicolon
\KwIn{A sequence of integers $\langle a_1, a_2, \ldots, a_n \rangle$}
\KwOut{The index of first location witht he same value as in a previous location in the sequence}
$location \gets 0$\;
$i \gets 2$\;
\While{$i \leq n \land location = 0$}{
  $j \gets 1$\;
  \While{$j < i \land location = 0$}{
    % The "l" before the If makes it so it does not expand to a second line
    \lIf{$a_i = a_j$}{
      $location \gets i$\;
    }
    \lElse{
      $j \gets j + 1$\;
    }
  }
  $i \gets i + 1$\;
}
\Return{location}\;
\caption{{\sc FindDuplicate2}}
\label{algo:duplicate2}
\end{algorithm}

\end{document}
